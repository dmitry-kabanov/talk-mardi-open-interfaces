\documentclass[10pt, aspectratio=169, progressbar=frametitle]{beamer}
\usetheme[numbering=fraction]{metropolis}

% -----------------------------------------------------------------------------
% Fonts
\usepackage[utf8]{inputenc}
\usepackage[defaultsans,scale=1.0]{lato}
\usepackage{fourier}
\usepackage[scale=1.05]{inconsolata}
\usepackage{microtype}
\usefonttheme[onlymath]{serif}

% -----------------------------------------------------------------------------
% Graphics
\usepackage{graphicx}
\usepackage{pgfplots}
\usepackage{tikz}
\usepackage[beamer,customcolors]{hf-tikz}
\usetikzlibrary{arrows.meta, calc, positioning, shapes.geometric}
\usepackage[dvipsnames]{xcolor}
\usepackage{tikz}
\usetikzlibrary{arrows.meta, calc, positioning, shapes.geometric}

% General colors
\definecolor{MaRDI-blue}{RGB}{0,94,170}
\definecolor{MaRDI-orange}{RGB}{208,102,43}
\colorlet{MaRDI-grey}{black!40} % 40% black
\colorlet{MaRDI-gray}{MaRDI-grey}

% Light colors
\colorlet{MaRDI-light-blue}{MaRDI-blue!60}
\colorlet{MaRDI-blue-light}{MaRDI-light-blue}
\colorlet{MaRDI-lightest-blue}{MaRDI-blue!40}
\colorlet{MaRDI-blue-lightest}{MaRDI-lightest-blue}

\tikzset{%
  every path/.style={thick},
    edge/.style={%
        ->,
        >={Stealth[scale=1]},
    },
    usercode/.style={%
        rectangle,
        fill=MaRDI-orange,
        draw=black, thick,
        minimum width=3em,
        rounded corners=2pt,
        text=white,
        font=\footnotesize \sffamily,
        text centered,
    },
    lib/.style={%
        ellipse,
        fill=MaRDI-gray,
        minimum width=.12\linewidth,
        minimum height=.04\linewidth,
        rounded corners=2pt,
        text=white,
        font=\footnotesize \sffamily,
        text centered,
    },
    solver/.style={%
        rectangle,
        fill=MaRDI-blue,
        draw=black, thick, dashed,
        text=white,
        font=\footnotesize \sffamily,
        text centered,
        minimum width=3em,
        rounded corners=2pt,
    },
    grid/.style={%
      gray!20,
    }
}


\graphicspath{{./assets}}

% -----------------------------------------------------------------------------
% Define useful colors.
\colorlet{black}{black!50!gray}
\colorlet{green}{green!50!gray}
\colorlet{blue}{blue!50!gray}

\colorlet{CBlue}{blue!15}
\colorlet{CBlueD}{blue!95!gray}
\colorlet{CRed}{red!15}
\colorlet{CRedD}{red!65!gray}
\colorlet{CGreenD}{green!60!gray}
\colorlet{COrange}{orange}

% -----------------------------------------------------------------------------
\setbeamersize{text margin left=0.5cm, text margin right=0.5cm}

% -----------------------------------------------------------------------------
% Title info
\title{Open Interfaces for Scientific Computing}
\author{Dmitry I. Kabanov, Ren\'e Fritze, Stephan Rave, Mario Ohlberger}
\institute{University of Münster}
\date{9 November 2023}

% -----------------------------------------------------------------------------
\begin{document}
\maketitle

\begin{frame}{Typical situation in scientific computing}
  \begin{itemize}
    \item Student Floyd is mostly proficient in Python
    \item However, for the new project, he needs to run a Monte-Carlo
          simulation on a forward model implemented as a C solver
    \item Which means that he needs to write bindings to invoke the solver
    \item He learns about another solver (in another language) and wants to try it
    \item There are \alert{two challenges:} different languages and different interfaces
  \end{itemize}
\end{frame}

\begin{frame}{Traditional way}
  \begin{minipage}{0.45\textwidth}
    \begin{itemize}
      \item For a solver in C provide bindings to Python, Julia, R, \dots
      \item Repeat for a set of $I$ implementations of algorithms and $L$ languages
      \item It leads to $I \times L$ amount of work if we want to provide
            bindings for each algorithm in each language
    \end{itemize}
  \end{minipage}\hfill
  \begin{minipage}{0.45\textwidth}
    \begin{tikzpicture}
  \node [usercode] (user_py) {Python user};
  \node [usercode] (user_c) {C user};
  \node [usercode] (user_jl) {Julia user};

  \node []
\end{tikzpicture}

  \end{minipage}
\end{frame}

\begin{frame}{Another solution}
  To solve both of these problems:
  \begin{itemize}
    \item provide automatic bindings between different languages
    \item provide generic interfaces for typical numerical problems
    \item Examples of generic interfaces are \texttt{scipy.optimize} in Python
          and \texttt{DifferentialEquations.jl} in Julia
          for solving differential equations
  \end{itemize}
\end{frame}

\begin{frame}{Open Interfaces subproject of the MaRDI project
  (\url{mardi4nfdi.de})}
  \begin{minipage}{0.45\textwidth}
    To solve both problems simultaneously:
    \begin{itemize}
      \item Provide a mediator library \texttt{liboif} that handles
            data marshalling between different languages and provides a set
            of generic interfaces for typical numerical problems
      \item Each interface allows for many implementations
      \item For $I$ implementations and $L$ languages leads to $I + L$
            amount of work
    \end{itemize}
  \end{minipage}\hfill%
  \begin{minipage}{0.50\textwidth}
    \input{tikz/oif_bindings.tex}
  \end{minipage}
\end{frame}

\begin{frame}{Software architecture}
  \begin{center}
    \includegraphics[scale=0.09]{arch.png}
  \end{center}
\end{frame}

\begin{frame}{Currently supported types, languages, problems}
  \begin{itemize}
    \item Types:
      \begin{itemize}
        \item \texttt{OIF\_F64} (64-bit floating-point numbers, a.k.a. C \texttt{double}),
        \item \texttt{OIF\_ARRAY\_F64} (arrays of doubles)
        \item \texttt{OIF\_CALLBACK} user-defined functions
          (e.g., right-hand side functions of differential equations)
      \end{itemize}
    \item Languages: Python, C. For C bindings, special structure \texttt{OIFArrayF64} and auxiliary
      functions are implemented to conveniently pass arrays
      along with their shape
    \item Problems
      \begin{itemize}
        \item quadratic equations: $ax^2 + bx + c = 0$
        \item systems of linear algebraic equations: $Ax = b$
        \item initial value problems: $y'(t) = f(t, y),\ y(t_0) = y_0$
      \end{itemize}
  \end{itemize}
\end{frame}

\begin{frame}[standout]
  \centering
  \Huge
  Thank you!
\end{frame}

\end{document}
